\customchapter{RESUMEN}

\begin{comment}
La reidentificación de personas (Person Re-ID) constituye un área de investigación fundamental dentro de la visión por computadora, cuyo objetivo es reconocer a un mismo individuo en múltiples cámaras o entornos sin depender exclusivamente de atributos faciales. Esta tarea resulta especialmente relevante en contextos de seguridad y vigilancia, donde desafíos como las variaciones de iluminación, pose, vestimenta y ángulos de cámara dificultan una identificación confiable. Si bien los enfoques supervisados han demostrado un desempeño competitivo, su fuerte dependencia de grandes volúmenes de datos etiquetados restringe su escalabilidad y aplicabilidad en escenarios reales.

La presente tesis investiga el aprendizaje autosupervisado como paradigma alternativo para la reidentificación de personas. Los métodos autosupervisados aprovechan las estructuras intrínsecas de los datos —mediante estrategias como contrastive learning y el preentrenamiento con transformers— para obtener representaciones discriminativas sin necesidad de anotaciones manuales. Este enfoque busca superar las limitaciones del aprendizaje supervisado, reducir los costos de etiquetado y mejorar la capacidad de generalización en entornos de vigilancia heterogéneos.

La investigación comprende una revisión sistemática de los avances recientes en Person Re-ID autosupervisado, el desarrollo de un marco experimental sobre datasets de referencia y una evaluación del desempeño de los modelos en términos de precisión, robustez y transferibilidad. Asimismo, se examinan los desafíos abiertos y se plantean direcciones futuras orientadas al diseño de modelos escalables, eficientes y adaptables. De este modo, este estudio contribuye a posicionar el aprendizaje autosupervisado como una alternativa prometedora y viable para los sistemas de monitoreo inteligente. \\
\end{comment}

El reconocimiento de marcha (\textit{Gait Recognition}) se ha consolidado como una modalidad biométrica relevante para la reidentificación de personas, al aprovechar patrones de movimiento únicos e independientes de la apariencia física. Sin embargo, los enfoques supervisados presentan una fuerte dependencia de datos etiquetados, mientras que los métodos auto-supervisados (\textit{Self-Supervised Learning}, SSL) carecen de la precisión necesaria para distinguir identidades similares. Esta investigación propone el desarrollo y evaluación de un modelo híbrido que combine la capacidad de generalización del aprendizaje auto-supervisado con la precisión del aprendizaje supervisado, buscando un equilibrio entre ambos paradigmas. El modelo se entrena en dos fases complementarias con el objetivo de mejorar la robustez y reducir la dependencia de anotaciones manuales. Los resultados esperados apuntan a demostrar la viabilidad de este enfoque para desarrollar sistemas de reidentificación de personas más escalables, precisos y aplicables a escenarios del mundo real.


\noindent \textbf{Palabras clave:} \\
Re-identificación de personas; Reconocimiento de marcha; Aprendizaje auto-supervisado; Aprendizaje supervisado; Modelo híbrido.

