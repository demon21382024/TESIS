\customchapter{ABSTRACT} 

\begin{comment}
\begin{center}
\large \vspace{-1.5cm} \textbf{PERSON RE-IDENTIFICATION USING SELF-SUPERVISED LEARNING}
\end{center}
\end{comment}

\begin{comment}
Person re-identification (Re-ID) constitutes a fundamental research area within computer vision, whose objective is to recognize the same individual across multiple cameras or environments without relying exclusively on facial attributes. This task is particularly relevant in security and surveillance domains, where challenges such as variations in illumination, pose, clothing, and camera viewpoints hinder reliable identification. Although supervised approaches have demonstrated competitive performance, their strong dependence on large-scale labeled datasets restricts their scalability and applicability in real-world scenarios.

This thesis investigates self-supervised learning as an alternative paradigm for Person Re-ID. Self-supervised methods leverage intrinsic structures of data—through strategies such as contrastive learning and transformer-based pretraining—to derive discriminative feature representations without the need for manual annotations. This approach seeks to address the limitations of supervised learning, reduce annotation costs, and improve generalization across heterogeneous surveillance environments.

The research comprises a systematic review of recent advances in self-supervised Person Re-ID, the development of an experimental framework on benchmark datasets, and an evaluation of model performance in terms of accuracy, robustness, and transferability. Moreover, it examines the open challenges and outlines future directions toward the design of scalable, efficient, and adaptable models. By doing so, this study contributes to positioning self-supervised learning as a promising and viable approach for intelligent monitoring systems. \\
\end{comment}
Gait recognition has become a relevant biometric modality for person re-identification, leveraging unique motion patterns that are independent of physical appearance. However, supervised approaches show a strong dependence on labeled data, while self-supervised learning (SSL) methods lack the precision required to distinguish between similar identities. This research proposes the development and evaluation of a hybrid model that combines the generalization capability of self-supervised learning with the accuracy of supervised learning, seeking a balance between both paradigms. The model is trained in two complementary phases with the goal of improving robustness and reducing dependence on manual annotations. The expected results aim to demonstrate the feasibility of this approach for developing person re-identification systems that are more scalable, accurate, and applicable to real-world scenarios.

\noindent \textbf{Keywords:} \\
Person re-identification; Gait recognition; Self-supervised learning; Supervised learning; Hybrid model.

