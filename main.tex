%% --------------------------------------------------------------------
%% Template UTEC Tesis
%% --------------------------------------------------------------------

% Este template ha sido modificado y actualizado por Eduardo Castro y Roosevelt Ubaldo en base a lo trabajado por Víctor Murray, Oscar Ramos y Juan Carlos Barbaran.
% Última actualización: Mayo, 2022

\documentclass[a4paper,12pt,oneside]{tesisutec}

\selectlanguage{spanish}

%% Paquetes
\usepackage[utf8]{inputenc}
\usepackage[round,authoryear]{natbib}
% Libreria de idioma
\usepackage[spanish]{babel}
% Libreria para posicionamiento
\usepackage{float}
% Librerias para insertar codigos
\usepackage[spanish,onelanguage,ruled,vlined]{algorithm2e}
\usepackage{verbatim} 
% Librería para hipervínculos
\usepackage{hyperref}
 % Librería necesaria para arreglar el orden de referencias en overleaf.com
\usepackage{notoccite}
\usepackage{tikz}
\usetikzlibrary{positioning, shapes, arrows.meta}
% Incluir acá los paquetes adicionales que deseas. 
\usepackage{comment}
\usepackage{booktabs}
\usepackage{tabularx}

% Ubicación de los imágenes.
\graphicspath{images/}

\makeatletter
% Reinsert missing \algbackskip
\def\algbackskip{\hskip-\ALG@thistlm}
\makeatother

\hypersetup{urlcolor=blue, colorlinks=true}

% ------------------------------

\begin{document}

\frontmatter

\department{CIENCIA DE LA COMPUTACIÓN}
\degree{bachiller }
\major{Ciencia de la Computación}

\title{UN ENFOQUE HÍBRIDO: GAIT-BASED PERSON RE-ID}

\author{
  CANTO VIDAL, HAROLD ALEXIS VICTOR \orcid{0000-0000-0000-0000} \\
  TORRES FARFÁN, JUAN MANUEL  \orcid{1111-1111-1111-1111}
}

\supervisor{CAYLLAHUA CAHUINA, EDWARD JORGE YURI \orcid{0000-0000-0000-0000}} % Es obligatorio agregar ORCID del asesor

\date{2025}

\maketitle

\setstretch{1.5}

% %% ============================================================================
%  Dedicatoria
%% ============================================================================

\dedicatory{
xxxxxxxxxxxxxxxxxxxxxxxxxxxxxxxxxxxxxxxxxxxxxxxxxxx
xxxxxxxxxxxxxxxxxxxxxxxxxxxxxxxxxxxxxxxxxxxxxxxxxxx
xxxxxxxxxxxxxxxxxxxxxxxxxxxxxxxxxxxxxxxxxxxxxxxxxxx
xxxxxxxxxxxxxxxxxxxxxxxxxxxxxxxxxxxxxxxxxxxxxxxxxxx
xxxxxxxxxxxxxxxxxxxxxxxxxxxxxxxxxxxxxxxxxxxxxxxxxxx
xxxxxxxxxxxxxxxxxxxxxxxxxxxxxxxxxxxxxxxxxxxxxxxxxxx
}

% \input{encabezados/agradecimientos}

\tableofcontents

\newpage

\listoftables

\newpage

\listoffigures

\addtocontents{toc}{\vspace{1.5em}}

%% ============================================================================
\mainmatter
\pagestyle{fancy}

\customchapter{RESUMEN}

\begin{comment}
La reidentificación de personas (Person Re-ID) constituye un área de investigación fundamental dentro de la visión por computadora, cuyo objetivo es reconocer a un mismo individuo en múltiples cámaras o entornos sin depender exclusivamente de atributos faciales. Esta tarea resulta especialmente relevante en contextos de seguridad y vigilancia, donde desafíos como las variaciones de iluminación, pose, vestimenta y ángulos de cámara dificultan una identificación confiable. Si bien los enfoques supervisados han demostrado un desempeño competitivo, su fuerte dependencia de grandes volúmenes de datos etiquetados restringe su escalabilidad y aplicabilidad en escenarios reales.

La presente tesis investiga el aprendizaje autosupervisado como paradigma alternativo para la reidentificación de personas. Los métodos autosupervisados aprovechan las estructuras intrínsecas de los datos —mediante estrategias como contrastive learning y el preentrenamiento con transformers— para obtener representaciones discriminativas sin necesidad de anotaciones manuales. Este enfoque busca superar las limitaciones del aprendizaje supervisado, reducir los costos de etiquetado y mejorar la capacidad de generalización en entornos de vigilancia heterogéneos.

La investigación comprende una revisión sistemática de los avances recientes en Person Re-ID autosupervisado, el desarrollo de un marco experimental sobre datasets de referencia y una evaluación del desempeño de los modelos en términos de precisión, robustez y transferibilidad. Asimismo, se examinan los desafíos abiertos y se plantean direcciones futuras orientadas al diseño de modelos escalables, eficientes y adaptables. De este modo, este estudio contribuye a posicionar el aprendizaje autosupervisado como una alternativa prometedora y viable para los sistemas de monitoreo inteligente. \\
\end{comment}

El reconocimiento de marcha (\textit{Gait Recognition}) se ha consolidado como una modalidad biométrica relevante para la reidentificación de personas, al aprovechar patrones de movimiento únicos e independientes de la apariencia física. Sin embargo, los enfoques supervisados presentan una fuerte dependencia de datos etiquetados, mientras que los métodos auto-supervisados (\textit{Self-Supervised Learning}, SSL) carecen de la precisión necesaria para distinguir identidades similares. Esta investigación propone el desarrollo y evaluación de un modelo híbrido que combine la capacidad de generalización del aprendizaje auto-supervisado con la precisión del aprendizaje supervisado, buscando un equilibrio entre ambos paradigmas. El modelo se entrena en dos fases complementarias con el objetivo de mejorar la robustez y reducir la dependencia de anotaciones manuales. Los resultados esperados apuntan a demostrar la viabilidad de este enfoque para desarrollar sistemas de reidentificación de personas más escalables, precisos y aplicables a escenarios del mundo real.


\noindent \textbf{Palabras clave:} \\
Re-identificación de personas; Reconocimiento de marcha; Aprendizaje auto-supervisado; Aprendizaje supervisado; Modelo híbrido.


\customchapter{ABSTRACT} 

\begin{comment}
\begin{center}
\large \vspace{-1.5cm} \textbf{PERSON RE-IDENTIFICATION USING SELF-SUPERVISED LEARNING}
\end{center}
\end{comment}

\begin{comment}
Person re-identification (Re-ID) constitutes a fundamental research area within computer vision, whose objective is to recognize the same individual across multiple cameras or environments without relying exclusively on facial attributes. This task is particularly relevant in security and surveillance domains, where challenges such as variations in illumination, pose, clothing, and camera viewpoints hinder reliable identification. Although supervised approaches have demonstrated competitive performance, their strong dependence on large-scale labeled datasets restricts their scalability and applicability in real-world scenarios.

This thesis investigates self-supervised learning as an alternative paradigm for Person Re-ID. Self-supervised methods leverage intrinsic structures of data—through strategies such as contrastive learning and transformer-based pretraining—to derive discriminative feature representations without the need for manual annotations. This approach seeks to address the limitations of supervised learning, reduce annotation costs, and improve generalization across heterogeneous surveillance environments.

The research comprises a systematic review of recent advances in self-supervised Person Re-ID, the development of an experimental framework on benchmark datasets, and an evaluation of model performance in terms of accuracy, robustness, and transferability. Moreover, it examines the open challenges and outlines future directions toward the design of scalable, efficient, and adaptable models. By doing so, this study contributes to positioning self-supervised learning as a promising and viable approach for intelligent monitoring systems. \\
\end{comment}
Gait recognition has become a relevant biometric modality for person re-identification, leveraging unique motion patterns that are independent of physical appearance. However, supervised approaches show a strong dependence on labeled data, while self-supervised learning (SSL) methods lack the precision required to distinguish between similar identities. This research proposes the development and evaluation of a hybrid model that combines the generalization capability of self-supervised learning with the accuracy of supervised learning, seeking a balance between both paradigms. The model is trained in two complementary phases with the goal of improving robustness and reducing dependence on manual annotations. The expected results aim to demonstrate the feasibility of this approach for developing person re-identification systems that are more scalable, accurate, and applicable to real-world scenarios.

\noindent \textbf{Keywords:} \\
Person re-identification; Gait recognition; Self-supervised learning; Supervised learning; Hybrid model.

 
\chapter{MOTIVACIÓN Y CONTEXTO} 

\section{Presentación del tema de investigación}

\begin{comment}
La re-identificación de personas (\textit{Person Re-ID}) es una tarea fundamental en el campo de la visión por computadora, cuyo objetivo es reconocer a un mismo individuo en diferentes cámaras o momentos. Entre las distintas modalidades biométricas, el reconocimiento de marcha (\textit{Gait Recognition}) destaca por utilizar la forma de caminar como rasgo distintivo, lo que lo hace menos sensible a cambios de vestimenta, pose o iluminación que los enfoques basados en la apariencia \citep{Purish2023GaitRecognition}.

Para entrenar estos sistemas, existen dos estrategias principales. La primera es el aprendizaje supervisado (\textit{supervised learning}), que depende de grandes volúmenes de datos etiquetados para alcanzar una alta precisión. La segunda es el aprendizaje autosupervisado (\textit{Self-Supervised Learning, SSL}), una alternativa reciente que permite aprender representaciones útiles a partir de videos sin etiquetas manuales, como lo demuestran trabajos de referencia \citep{Zheng2022GaitLU1M, Wang2025GaitForeMer}. Este paradigma ha demostrado ser especialmente valioso para aprovechar datos no estructurados y reducir la dependencia del etiquetado manual.
\end{comment}

La re-identificación de personas (\textit{Person Re-ID}) se posiciona entre una tarea en el campo de visión por computadora y las necesidades operativas de vigilancia, análisis forense y sistemas inteligentes. El objetivo no es sólo identificar personas sino mantener una asociación fiable de identidades a través de múltiples cámaras, condiciones de iluminación y lapsos temporales. Estos retos nos presentan problemas prácticos como las oclusiones parciales, cambios de pose y la gran variabilidad de apariencia en las personas. En consecuencia, la re-identificación de personas se convierte en una tarea donde la robustez de las representaciones y la capacidad del modelo para generalizar escenarios son tan importantes como su precisión nominal. Por tal, es común que se busquen mejoras arquitectónicas como entrategias de entrenamiento y normalización de dominios, que puedan aumentar la precisión bajo condiciones reales.

Dentro de las modalidades biométricas que trabajan bajo la re-identificación de personas, el reconocimiento de marcha (\textit{Gait Recognition}) aparece como un rasgo complementario con ventajas prácticas claves, ya que la marcha puede extraerse a distancia, funciona cuando la cara no es visible y suele ser relativamente estable ante cambios de la vestimenta, escencial en escenarios de larga distancia o con cámaras con baja resolución \citep{Purish2023GaitRecognition}. Sin embargo, trabajar bajo el reconocimiento de marcha exige capturar información temporal y estructural, como la dinámica corporal, secuencia de siluetas y patrones cíclicos; además de lidiar con variaciones por ángulo de visión, velocidad de desplazamiento, fondos complejos y ruido en la segmentación de siluetas. En consecuencia, esta modalidad biométrica nos permite diseñar representaciones que conserven la información temporal relevante y escalable en grandes escenarios. 

En este contexto, el aprendizaje supervisado (\textit{Supervised Learning}) es la vía más directa para alcanzar una alta precisión en tareas de re-identificación; las redes profundas entrenadas con pares o tripletas, pérdida de clasificación y conjuntos etiquetadas, permiten optimizar directamente la discriminación. No obstante, este paradigma presenta limitaciones prácticas como la necesidad de anotaciones y su recolección, y la tendencia al sobreajuste a condiciones que no representan todas las variables del mundo real. Además, los  modelos supervisados suelen perder eficacia cuando se despliegan en nuevos entornos. Esto evidencia estrategias que reduzcan la dependencia del etiquetado exhaustivo y mejore la transferencia entre dominios.

En contraste, el aprendizaje autosupervisado (\textit{Self Supervised Learning - SSL}) ofrece explorar grandes colecciones sin etiquetas, aprendiendo invaricancias útiles mediante \textit{pretext-tasks}. El reconocimiento de marcha puede utilizar características espacio-temporales que codifican la dinámica de marcha y robustez frente a diferentes vistas. Aquí es donde ambos enfoques de aprendizajes convergen, un enfoque híbrido que combine el pre-entrenamiento autosupervisado sobre grandes volúmenes y posteriormente realizar un ajuste supervisado sobre conjuntos anotados, buscando mejroar la generalización entre cámaras y recuperación de discriminantes finos necesarios para la re-identificación.

\section{Descripción de la situación problemática}

A pesar de sus avances, \textit{Gait Recognition} aún enfrenta desafíos relacionados con las limitaciones de ambas estrategias de aprendizaje. Los métodos supervisados ofrecen una alta precisión, pero su escalabilidad se ve comprometida por la necesidad de datos etiquetados en contextos reales, donde las variaciones visuales son casi infinitas. Y, por otro lado, el aprendizaje autosupervisado permite una mayor generalización y adaptabilidad, pero carece de una guía explícita que ayude al modelo a distinguir detalles sutiles entre individuos con patrones de caminata similares.

Para abordar estas deficiencias, investigaciones recientes han explorado modelos híbridos y arquitecturas más complejas. Por ejemplo, \citet{Kovacevic2021SelfAttentionGait} introduce mecanismos de auto-atención que mejora la capacidad del modelo para concentrarse en las regiones más relevantes del movimiento. No obstante, incluso estos enfoques avanzados no logran resolver completamente el problema, ya que siguen careciendo de un componente supervisado que refine la discriminación entre identidades.

En este contexto, el problema que esta tesis aborda radica en la necesidad de un marco de trabajo que combine de manera efectiva la capacidad de generalización del aprendizaje autosupervisado con la precisión del aprendizaje supervisado. Más que una simple combinación, se busca explorar una integración sinérgica entre ambos paradigmas, aprovechando la robustez de los modelos basados en atención y la guía estructurada del aprendizaje supervisado para mejorar la fiabilidad y adaptabilidad del reconocimiento de marcha en escenarios reales.

\section{Formulación del problema}

En \textit{Gait Recognition}, los métodos supervisados ofrecen alta precisión, pero dependen de grandes volúmenes de datos etiquetados, lo que limita su escalabilidad. Por otro lado, el aprendizaje autosupervisado reduce esa dependencia, aunque pierde capacidad para diferenciar identidades similares. Las propuestas híbridas actuales combinan ambos enfoques de forma parcial, sin una integración real. Por ello, esta investigación aborda la falta de un marco unificado que combine de manera efectiva la precisión del aprendizaje supervisado y la generalizacón del aprendizaje autosupervisado para mejorar la re-identificación de personas.

En \textit{Gait Recognition} no hay un marco que logre integrar de manera efectiva la capacidad de generalización del aprendizaje supervisado con la alta precisión discrimnativa del aprendizaje supervisado. Los métodos disponibles combinan ambos enfoques de forma parcial o secuencial, lo que limita su rendimiento en escenarios reales con variaciones significativas entre individuos. Por ello, el problema de la presente investigación consiste en diseñar y evalua un modelo híbrido que unifique ambos paradigmas para mejorar la diabilidad y la adaptabilidad del proceso de re-identificación.

\section{Objetivos de investigación}

\subsection*{Objetivo General}

Desarrollar y evaluar un modelo híbrido que combine el aprendizaje supervisado y autosupervisado para la re-identificación de personas mediante \textit{Gait Recognition}.

\subsection*{Objetivos Específicos}
\begin{enumerate}
    \item Analizar el estado del arte de los métodos supervisados, autosupervisados y combinados aplicados al \textit{Person Re-ID} mediante \textit{Gait recognition}, identificando sus principales limitaciones y tendencias recientes.
    \item Diseñar una arquitectura híbrida que integre un módulo de pre-entrenamiento autosupervisado con una etapa de ajuste fino supervisado, buscando un equilibrio entre generalización y precisión.
    \item Implementar el modelo propuesto utilizando un conjunto de datos no etiquetado de gran escala para el pre-entrenamiento y datasets etiquetados estándar para el entrenamiento supervisado.
    \item Evaluar el desempeño, bajo métricas de precisión, del enfoque híbrido frente a métodos puramente supervisados y puramente autosupervisados.
\end{enumerate}

\section{Justificación}

Este trabajo busca superar las limitaciones de los enfoques híbridos actuales en el reconocimiento de marcha. Aunque existen propuestas que combinan el aprendizaje supervisado y autosupervisado \citep{Li2020SemiSupGait}, la mayoría lo hace de manera secuencial, sin lograr una integración efectiva entre ambos paradigmas. Frente a ellos, la investigación propone un marco unificado que permita la interacción sinérgica entre las dos estrategias de aprendizaje, contribuyendo así al desarrollo de modelos más equilibrados entre precisión y capacidad de generalización.

En el plano práctico, esta propuesta tiene un impacto tangible al reducir la dependencia de grandes volúmenes de datos etiquetados, lo que disminuye los costos y la complejidad asociados a la implementación de sistemas de re-identificación a gran escala. Un modelo más generalizable, como los explorados en trabajos recientes \citep{Dou2023IdentitySeeking}, facilitaría la transición de los sistemas actuales desde entornos controlados de laboratorio hacia aplicaciones reales, aumentando su robustez y confiabilidad.

Finalmente, la justificación social radica en el potencial de este tipo de tecnologías para fortalecer la seguridad ciudadana, el monitoreo inteligente de espacios públicos y el análisis forense. Al mejorar la precisión y escalabilidad de los sistemas de reconocimiento de marcha, esta investigación contribuye al desarrollo de soluciones tecnológicas más efectivas y socialmente relevantes, capaces de apoyar la gestión de entornos urbanos de manera ética y responsable.

\section{Alcance y limitaciones / restricciones} %[opcional]

El alcance de esta investigación abarca el diseño, implementación y evaluación experimental de un diseño híbrido para la re-identificación de personas mediante \textit{Gait Recognition}. El estudio se limita al uso de datos visuales provenientes de conjuntos de datos públicos y se centra en la comparación del modelo propuesto frente a enfoques supervisados y autosupervisados, utilizando métricas estándar. \\
La investigación no contempla el desarrollo de un sistema de vigilancia en tiempo real ni la optimización para hardware específico.Asimismo, el entrenamiento de los modelos estarám condicionados por los recursos computacionales disponibles, lo que puede restringir la magnitud del estudio. 

\textit{Nota: El documento corresponde a una entrega parcial del trabajo de tesis, por lo que el alcance y las limitaciones  pueden estar sujetos a cambios en función del avance y los resultados experimentales obtenidos.}
\chapter{REVISIÓN CRÍTICA DE LA LITERATURA}

\begin{comment}
La reidentificación de personas es un problema de visión por computador cuyo objetivo es determinar si imágenes o secuencias, procedentes de distintas cámaras, pertenecen a la misma persona. La complejidad de esta tarea reside en la gran variabilidad intra-clase inducida por cambio de poses, oclusiones, iluminación o resolución, así como en la reducida variabilidad inter-clase que presentan con apariencia o vestimenta similar. Esta dualidad exige el aprendizaje de representaciones altamente discriminativas y robustas.
Históricamente, las soluciones supervisadas (Supervised Learning) han alcanzado altos rendimientos cuando existen anotaciones abundantes; sin embargo, en escenarios reales la obtención de etiquetas es costosa, sujeta a privacidad y poco escalable. En este contexto, los métodos que aplican aprendizaje autosupervisado (Self-Supervised Learning - SSL) emergen como una paradigma prometedor para aprovechar grandes colecciones no etiquetadas y aprender representaciones transferibles y robustas sin dependencia directa de anotaciones humanas previas.

En este primer capítulo buscamos sintetizar y comparar cómo distintos métodos que aplican Self-Supervised Learning diseñan señales de aprendizaje (pretext tasks, pérdidas y arquitecturas) para capturar características discriminativas en Re-ID. Así mismo, evaluar la evidencia en datasets, protocolos y métricas que respalden sus afirmaciones y reproducibilidad. Y, por último, identificaremos las limitaciones comunes de estos proyectos como puede ser el coste computacional, dependencia de colecciones no etiquetadas u oclusiones.
\end{comment}

\textit{Person Re-ID} tiene como objetivo determinar si imágenes o secuencias procedentes de distintas cámaras, pertenecen a la misma persona. La complejidad de esta tarea radica en la variabilidad intra-clase inducida por cambios de pose, oclusiones, iluminación o resolución; así como en la reducida variabilidad inter-clase que presentan individuos con apariencia o vestimenta similar. Esta dualidad exige el aprendizaje de representaciones altamente discriminativas y robustas.

Históricamente, las soluciones basadas en aprendizaje supervisado (\textit{Supervised Learning}) han alcanzado altos rendimientos cuando existen anotaciones abundantes y de alta calidad. Sin embargo, en escenarios reales la obtención de etiquetas es costosa, sujeta a restricciones de privacidad y poco escalable. En este contexto, los métodos de aprendizaje autosupervisado (\textit{Self-Supervised Learning, SSL}) nacen como un paradigma prometedor para aprovechar grandes colecciones no etiquetadas y aprender representaciones transferibles y robustas sin dependencia directa de anotaciones humanas previas.

Aunque este trabajo se centra en el reconocimiento por marcha (\textit{Gait Recognition}), la literatura de \textit{Person Re-ID} basada en apariencia y en pre-entrenamiento autosupervisado aporta técnicas clave como el aprendizaje contrastivo, enmascaramiento de imagen (\textit{Masked Image Modeling, MIM}), entre otras; que pueden ser enfocadas en \textit{Gait Recognition}. Por tanto, para la revisión de literatura, se incluyen y analizan métodos representativos de \textit{Person Re-ID} con el fin de identificar técnicas de pre-entrenamiento robustas; evaluar estrategias de aumento de datos y representación de partes corporales; y detectar brechas en la aplicación de aprendizaje autosupervisado.

Esta revisión crítica busca, en primer lugar, sintetizar y comparar cómo los distintos métodos que aplican aprendizaje autosupervisado diseñan señales de entrenamiento para capturar características discriminativas en \textit{Re-ID}. En segundo lugar, se evalúa la evidencia experimental reportada en términos de \textit{datasets} y métricas de desempeño. Finalmente, se discuten las limitaciones comunes observadas en la literatura, como el costo computacional, la dependencia de grandes volúmenes de datos no etiquetados y la sensibilidad a oclusiones o variaciones de postura.

\begin{comment}
\section{PersonViT: Large-scale Self-supervised Vision Transformer for Person Re-Identification (2024)}

PersonViT es una estrategia propuesta  de pre-entrenamiento self-supervised a gran escala para Re-ID basada en Vision Transformers que combina dos señales: DINO (self-distillation contrastiva con multi-crop) para conservar correspondencias globales y locales, y una pérdida de Masked Image Modeling (MIM) aplicada a parches que fuerza al modelo a aprender características locales finas sin particiones manuales. El pipeline se basa en pre-entrenar ViT sobre LUPerson (millones de imágenes no etiquetadas) con la combinación L = Ldino + Lmim, y después afinar supervisadamente con esquemas estándar de Re-ID (ID loss + triplet, BNNeck). La inclusión de MIM aporta ganancias significativas y el rendimiento escala con la fracción de LUPerson empleada, lo que evidencia la importancia del pre-entrenamiento masivo para ViT en Re-ID.

Las visualizaciones de atención muestran que PersonViT aprende correspondencias parciales útiles en casos de oclusión, y la variante con backbone ViT-B/16 alcanza resultados adecuados en MSMT17, Market1501, Duke y Occluded-Duke. No obstante, la propuesta tiene un coste computacional elevado, el pre-entrenamiento en LUPerson requiere recursos GPU considerables (grandes batches y muchas épocas), y el modelo pesado (ViT-B) amplifica requisitos de memoria y tiempo de inferencia. Además, la sensibilidad a la composición del conjunto de pre-entrenamiento y el ajuste fino de hiper-parámetros obligan a una experimentación extensa para producir resultados óptimos.


\section{A Self-Supervised Gait Encoding Approach with Locality-Awareness for 3D Skeleton Based Person Re-Identification (2021)}

Se presenta un enfoque autosupervisado dirigido a aprender representaciones de gait bajo secuencias de esqueletos 3D, con el objetivo de producir embeddings discriminativos para “Person Re-Identification”. La propuesta combina un esquema encoder-decoder (LSTM) con una tarea de pretexto de reverse sequence reconstruction, diseñada para forzar la captura de dependencias temporales, y un Locality-Aware Attention (LAS) que impone una preferencia por pesos de atención cercanos en el eje temporal, con el fin de retener información local relevante para el gait. A partir de los vectores de contexto de atención se construyen los Attention-based Gait Encodings (AGEs) y se afina la representación mediante “pérdida contrastiva” que explota el “locality inter-secuencia”  (Locality-Aware Contrastive Learning, LCL), donde secuencias temporalmente cercanas se tratan como positivas. De esta combinación resultan las CAGEs, representaciones que compiten con métodos skeleton-based y con algunos enfoques multimodales en benchmarks como BIWI, KS20 y KGBD.
Entre las principales fortalezas del método figuran la capacidad de aprender sin etiquetas, la explotación explícita de la estructura temporal y la flexibilidad de operar con esqueletos procedentes de sensores 3D o estimadores de pose sobre RGB. 

Sin embargo, el enfoque tiene limitaciones relevantes para su adopción práctica, los datasets empleados son pequeños y controlados, lo que plantea dudas de generalización a escenarios ; la calidad final depende fuertemente de la precisión del esqueleto (los errores del pose estimator degradan la representación); y las variantes más completas (ej. combinaciones de pretextos) aumentan notablemente el coste computacional y la dimensionalidad de las features, además de requerir batches estructurados para el contraste, lo que incrementa la memoria GPU durante el entrenamiento.	

\section{Identity-Seeking Self-Supervised Representation Learning for Generalizable Person Re-Identification (2023)}

El trabajo introduce \textit{Identity-seeking Self-supervised Representation learning (ISR)}, un método autosupervisado para la reidentificación de personas (Person Re-ID) orientado a la  generalización entre dominios. A diferencia de los enfoques previos de DG Re-ID que requieren datos etiquetados costosos y limitados, ISR entrena sobre videos a gran escala sin anotaciones. La propuesta construye pares positivos entre imágenes de distintos fotogramas modelando la asociación de instancias como un problema de \textit{maximum-weight bipartite matching}. Además, incorpora una \textit{reliability-guided contrastive loss} para reducir el impacto de 
pares positivos ruidosos y asegurar que el aprendizaje se base en asociaciones confiables.  

Los experimentos muestran que ISR escala de manera casi lineal con la cantidad de datos, lo que permite aprovechar grandes volúmenes sin incurrir en costos prohibitivos. En benchmarks estándar, el método logra resultados sobresalientes: \textbf{87.0\% Rank-1 en Market-1501} y  \textbf{56.4\% Rank-1 en MSMT17} sin necesidad de anotaciones ni \textit{fine-tuning}, superando  en hasta \textbf{19.5\%} a los mejores enfoques supervisados de generalización. Bajo el esquema de \textit{pre-training → fine-tuning}, ISR establece nuevo \textit{state-of-the-art} en MSMT17 con \textbf{88.4\% Rank-1}.  

Sin embargo, el método presenta ciertos retos: depende de la correcta construcción de pares \textit{inter-frame}, lo que puede ser sensible a ruido en escenarios de vigilancia complejos, y aunque su escalabilidad es positiva, su desempeño podría contrastarse con otras  estrategias autosupervisadas de menor complejidad.

\section{Using Optical Flow Consistency to Improve Segmentation Stability in Long-Duration Surveillance Video (2025)}

Los autores abordan el problema de la inestabilidad temporal en la segmentación semántica de video, un desafío crítico en escenarios de vigilancia prolongada. Aunque los métodos cuadro a cuadro han alcanzado gran precisión, suelen producir "flickering" e inconsistencias a lo largo de la secuencia. Para resolverlo, proponen un marco de entrenamiento end-to-end que combina un backbone moderno de segmentación con una red eficiente de estimación de flujo óptico. La contribución central es una pérdida de consistencia autosupervisada basada en flujo, que utiliza vectores de movimiento para reforzar la coherencia temporal entre cuadros consecutivos sin sacrificar precisión individual. En evaluaciones sobre datasets complejos como MOSE, el método mejora la estabilidad de la segmentación en hasta 15.7\%, alcanzando un nuevo estado del arte en métricas como mIoU y mTC.

Por otro lado, el trabajo reconoce limitaciones y compromisos: la dependencia del flujo óptico introduce sobrecosto computacional que puede afectar la aplicabilidad en tiempo real, especialmente en dispositivos de baja potencia. Además, el rendimiento depende directamente de la calidad del flujo, que se degrada en condiciones adversas como baja iluminación, mal clima o desenfoque extremo. Esto abre la puerta al desarrollo de métodos libres de flujo, que podrían complementar o sustituir la propuesta en escenarios más hostiles. Finalmente, persiste el reto de la reidentificación de objetos tras largas desapariciones, lo que señala la necesidad de integrar mecanismos avanzados de memoria o tracking a largo plazo.

\section{Self-Supervised Low-FPS Multiple Object Tracking for UAV: Introducing 5 FPS Benchmark and Methodological Advancements (2025)}

Los autores exploran el problema del \textit{multiple object tracking} (MOT) en videos capturados por vehículos aéreos no tripulados (UAVs), donde factores como cambios frecuentes de ángulo, transiciones de escena y variaciones de apariencia complican el seguimiento. La situación se agrava en escenarios de baja tasa de cuadros (low-FPS), comunes por las limitaciones computacionales al procesar múltiples flujos de video en paralelo, lo que produce grandes desplazamientos entre cuadros y afecta el rendimiento de los trackers. Para reducir la dependencia de anotaciones manuales costosas y propensas a error, se investigan técnicas de aprendizaje autosupervisado, junto con mecanismos de memoria a largo plazo para mejorar la reidentificación de objetos tras amplias brechas temporales. El trabajo compara enfoques basados en similaridad quasi-densa, grafos y pseudo-etiquetado mediante modelos maestro, empleando detectores preentrenados como YOLO en datasets militares. Además, se introduce un benchmark novedoso diseñado para imágenes militares a 5 FPS, que permite evaluar de forma integral distintos métodos bajo restricciones realistas.

En cuanto a limitaciones, se destaca que las predicciones ruidosas afectan las métricas de MOT y aún representan un desafío clave. Asimismo, el bajo número de cuadros por segundo restringe la granularidad temporal disponible, lo que plantea la necesidad de trackers más robustos y eficientes en entornos con recursos limitados. El estudio abre camino hacia el diseño de métodos de tracking especializados para escenarios militares y aplicaciones críticas donde la anotación exhaustiva y el procesamiento en tiempo real no son factibles.

\section{UCM-VeID V2: A Richer Dataset and A Pre-training Method for UAV Cross-Modality Vehicle Re-Identification (2025)}

El trabajo aborda el problema del \textit{Vehicle Cross-Modality Re-Identification} (VT-ReID), cuyo objetivo es permitir la identificación de vehículos tanto en modalidad RGB como infrarroja (IR), favoreciendo aplicaciones de vigilancia continua día y noche. Una de las principales limitaciones en este campo es la escasez de datasets amplios y representativos, así como el sesgo introducido por preentrenamientos en ImageNet que generan \textit{modality bias training} (MBT). Para enfrentar estos retos, los autores presentan un nuevo benchmark llamado \textbf{UCM-VeID V2}, que incrementa de manera sustancial el volumen de datos y mejora múltiples aspectos de las capturas. Además, proponen un método de preentrenamiento autosupervisado denominado \textbf{Patch-Mixed Self-supervised Learning (PMSL)}, diseñado para aprender representaciones invariantes entre modalidades. PMSL integra tres componentes clave: la reconstrucción de imágenes con mezcla de parches (\textit{Patch-Mixed Image Reconstruction, PMIR}), el aprendizaje adversarial de discriminación de modalidad (\textit{MDAL}) y el contraste de clústeres con aumento de modalidad (\textit{MACC}).

Los experimentos demuestran que la combinación de UCM-VeID V2 y PMSL no solo mitiga el sesgo de modalidad, sino que también mejora la capacidad discriminativa de los modelos en escenarios reales. Sin embargo, aún persisten retos relacionados con la complejidad de las estrategias adversariales y contrastivas, así como la necesidad de validar la escalabilidad del método en contextos operativos más amplios y variados.
\section{Taxonomía del Estado del Arte}
\end{comment}

\begin{comment}
A continuación se presenta una taxonomía organizada por paradigmas de aprendizaje —\textbf{Self-Supervised}, \textbf{Supervised} y \textbf{Unsupervised}— aplicada a Person Re-Identification y Gait Recognition. Para cada rama se indica una metodología representativa seguida de una referencia y el año entre paréntesis (ej.: (\cite{Clave}, año)).
\end{comment}

\begin{comment}
\section{Self-Supervised Learning (SSL)}

\begin{itemize}
  \item \textbf{Contrastive learning} (PASS / instance/contrastive schemes) (\cite{Zheng2022PASS}, 2022; \cite{Dou2023IdentitySeeking}, 2023).
  \item \textbf{Masked reconstruction / Masked Sequence Modeling} (pretexto reconstructivo tipo MAE adaptado a gait) (\cite{Han2021LocalitySGE}, 2021; \cite{Wang2025GaitForeMer}, 2025).
  \item \textbf{Temporal forecasting / motion prediction} (predecir frames o dinámica para aprender representación) (\cite{Zheng2021SelfGait}, 2021; \cite{Duan2024FSGait}, 2024).
  \item \textbf{Skeleton / Graph-based SSL} (aprendizaje autosupervisado sobre esqueletos y grafos) (\cite{Han2021SMSGE}, 2021).
  \item \textbf{Generative / restoration-oriented SSL} (restauración o distillation orientada a Re-ID) (\cite{Li2025OrientedKD}, 2025).
  \item \textbf{Large-scale unlabeled pretraining (datasets para SSL)} (corpus no etiquetado para pre-entrenamiento de gait) (\cite{Zheng2022GaitLU1M}, 2022; \cite{Liu2023GaitBenchmark}, 2023).
\end{itemize}

\section{Supervised Learning}

\begin{itemize}
  \item \textbf{Metric learning / Triplet and variants} (pérdidas de triplet, soft-labels para Re-ID) (\cite{Lin2020Softened}, 2020; \cite{Guo2023TripletContrastive}, 2023).
  \item \textbf{CNN-based supervised models para gait} (desentrelazado identidad/covariables) (\cite{Li2020SemiSupGait}, 2020).
  \item \textbf{Transformer-based supervised models} (Vision Transformers aplicados a person Re-ID) (\cite{Huang2025PersonViT}, 2025; \cite{He2025TransReID}, 2025).
  \item \textbf{Part-aware / attention supervised approaches} (plantillas/residual attention para detalles discriminativos) (\cite{Yang2020DevilDetails}, 2020).
\end{itemize}

\section{Unsupervised Learning}

\begin{itemize}
  \item \textbf{Clustering-based unsupervised learning} (clustering iterativo para generar pseudo-etiquetas) (\cite{Lin2020Softened}, 2020; \cite{Wu2020Tracklet}, 2020).
  \item \textbf{Domain adaptation / multi-scene unsupervised strategies} (enfoques para generalización entre escenas) (\cite{Zhou2024VersReID}, 2024; \cite{Xu2023VILLS}, 2023).
  \item \textbf{Synthetic-to-real transfer / unsupervised sim2real} (usar datos sintéticos y adaptar a real) (\cite{Li2024SyntheticReID}, 2024).
  \item \textbf{Low-FPS / weak supervision / tracker-based unsupervised} (métodos auto y no supervisados para tracking y re-id en escenarios difíciles) (\cite{Smith2025LowFPS}, 2025; \cite{Joshua2025OpticalFlow}, 2025).
\end{itemize}

\section{Aprendizajes Híbridos y Métodos Transversales}

\begin{itemize}
  \item \textbf{SSL $\rightarrow$ Fine-tuning supervisado} (pre-entrenamiento self-supervised seguido de fine-tuning supervisado en benchmarks) (\cite{Dou2023IdentitySeeking}, 2023; \cite{Huang2025PersonViT}, 2025).
  \item \textbf{Knowledge distillation orientada a Re-ID} (teacher-student para robustecer representaciones) (\cite{Li2025OrientedKD}, 2025).
  \item \textbf{Cycle / association based learning} (asociaciones en ciclos diferenciables para re-ID sin etiquetas) (\cite{Gao2025CycAs}, 2025).
  \item \textbf{Cross-modality / multi-platform pretraining} (pre-entrenamiento y transferencia entre modalidades, e.g., UAV + CCTV) (\cite{Liu2025UCMVeID}, 2025; \cite{Rao2024MSFFT}, 2024).
\end{itemize}

\section{Tabla de resumen}
\begin{table}[H]
    \centering
    \caption{Resumen taxonómico: metodología y referencia representativa.}
    \begin{tabular}{p{6.5cm} p{6.5cm}}
    \hline
    \textbf{Metodología} & \textbf{Paper representativo (clave, año)} \\ \hline\hline
    Contrastive SSL & \cite{Dou2023IdentitySeeking}, 2023 \\ 
    Masked reconstruction / Masked Sequence Modeling & \cite{Han2021LocalitySGE}, 2021 \\ 
    Temporal forecasting / predictive SSL & \cite{Zheng2021SelfGait}, 2021 \\ 
    Graph / Skeleton SSL & \cite{Han2021SMSGE}, 2021 \\ 
    Large-scale unlabeled pretraining (datasets) & \cite{Zheng2022GaitLU1M}, 2022 \\ 
    Transformer-based supervised models & \cite{Huang2025PersonViT}, 2025 \\ 
    Metric learning / Triplet & \cite{Lin2020Softened}, 2020 \\ 
    Clustering-based unsupervised & \cite{Wu2020Tracklet}, 2020 \\ 
    Synthetic-to-real transfer & \cite{Li2024SyntheticReID}, 2024 \\ 
    Cycle association (SSL) & \cite{Gao2025CycAs}, 2025 \\ \hline
    \end{tabular}
    \label{tab:taxonomy_summary}
\end{table}
\end{comment}

\section{Aprendizaje Autosupervisado}

Los trabajos basados en aprendizaje auto-supervisado muestran una diversificación de pretext-tasks orientadas a capturar distintas propiedades útiles para Re-ID. Las estrategias basadas en \textit{contrastive learning}, enfoques que construyen pares a partir de frames y tracklets (\cite{Zheng2022PASS}; \cite{Dou2023IdentitySeeking}), buscan explotar la continuidad temporal de videos para obtener invariancias útiles entre vistas y son representadas por métodos como \textit{Identity-Seeking (ISR)}, que emplea este emparejamiento y pérdida contrastiva por confiabilidad para mitigar pares ruidosos. Sin embargo, puede llegar depender fuertemente de la calidad del extractor de características y de la selección temporal de pares.

Así mismo, se investiga también sobre enmascaramiento \textit{(MIM / Masked Sequence Modeling)} y tareas reconstructivas que han sido adaptadas para extraer características locales sin requerir alineación manual; como es el caso de PersonViT, que incorpora MIM para obtener representaciones locales de grano fino y reducir dependencia de divisiones manuales de partes (\cite{Han2021LocalitySGE}; \cite{Wang2025GaitForeMer}), lo que es muy relevante cuando se trabaja con silhouettes o secuencias de marcha donde la alineación es difícil, aunque su alto costo computacional y modelos complejos de pre-entrenamiento (LUPerson - Utilizado también en \cite{Zheng2022PASS}; \cite{Li2025OrientedKD}; \cite{Xu2023VILLS}) son limitaciones claras.

Otros métodos aprovechan modelos geométricos, esqueletos y grafos, donde métodos como SM-SGE (\cite{Han2021SMSGE}) y CAGES (\cite{Han2021LocalitySGE}) aplican tareas de reconstrucción \textit{multi-scale} y pérdidas contrastivas con conciencia de localidad para conservar relaciones intra-secuencia y estructura corporal. Estos enfoques son particularmente prometedores para \textit{Gait Recognition}, ya que explotan explícitamente la dinámica corporal, pero su desempeño queda restringido por la calidad y disponibilidad de esqueletos 3D y por el tamaño reducido de \textit{datasets} disponibles.

\section{Aprendizaje supervisado}

Los trabajos supervisados mantienen una línea de referencia para medir el rendimiento, enfoques basados en metric learning (triplet y variantes con soft-labels) se mantienen como estándar para optimizar la separación identidad-interna y la discriminación entre clases cercanas, aunque sufren cuando las etiquetas son escasas o ruidosas y no generalizan bien entre escenas muy distintas (\cite{Lin2020Softened}; \cite{Guo2023TripletContrastive}). Estudios sobre pérdidas tipo triplet y sus variantes muestran robustez en benchmarks clásicos, pero exigen anotaciones costosas y estrategias de muestreo cuidadosas. 

Por otro lado, los CNNs tradicionales resultan competitivos en\textit{gait recognition} cuando se diseñan para separar identidad y covariables, pero la adopción creciente de Transformers y ViT-based supervised models (\cite{Huang2025PersonViT}; \cite{He2025TransReID}) evidencia una tendencia hacia \textit{backbones} que capturan relaciones a larga distancia y tokens; estos modelos tienden a obtener mejoras en escenarios donde hay suficiente data anotada.

Finalmente, los métodos \textit{part-aware} (\cite{Yang2020DevilDetails}) y basados en atención proveen discriminación a nivel de región, especialmente útil en \textit{gait recognition} si se busca resaltar los segmentos corporales por separado. No obstante, su dependencia en buenas detecciones y segmentaciones, y la posible saturación en datasets implican que las ganancias pueden ser modestas en escenarios reales.

\section{Aprendizaje no supervisado}

Dentro de los trabajos basados en aprendizaje no supervisado, Softened Similarity Learning (\cite{Lin2020Softened}) propone eliminar el clustering y suavizar etiquetas mediante distribuciones sobre imágenes similares, lo que amortigua errores de cuantización pero sigue siendo sensible a la medida de similitud y a hiperparámetros, demostrando que es posible aprender descriptores razonables sin anotaciones, aunque la estabilidad y la precisión final suelen estar por debajo de otros enfoques.

En contraste, hay métodos que explotan información temporal (tracklets) y asociación cíclica para aprender sin etiquetas (tracklet-based (\cite{Wu2020Tracklet}), los cuales reducen la dependencia de clustering y focalizan la señal de aprendizaje en la consistencia temporal, pero requieren datos con solapamiento temporal y sufren cuando la detección o el tracking son ruidosos. 

Adicionalmente, estrategias de synthetic-to-real (\cite{Zhang2024SyntheticReID}) aparece como una alternativa para superar la escasez de anotaciones, ya que la generación sintética permite controlar variaciones (vistas, ropa, oclusión); pero la adaptación al dominio real añade complejidad y la calidad de generación determina el éxito final.

\section{Aprendizajes híbridos}

Los métodos híbridos nos presentan una serie de nuevos enfoques como pre-entrenamiento SSL seguido de fine-tuning supervisado (\cite{Dou2023IdentitySeeking}; \cite{Huang2025PersonViT}), distillation orientada a Re-ID (\cite{Li2025OrientedKD}) y frameworks multi-escena (\cite{Liu2025UCMVeID}; \cite{Rao2024MSFFT}). Estos métodos permiten máximizar la generalización; por ejemplo, frameworks como VersReID (\cite{Zhou2024VersReID}) emplean pre-entrenamiento autosupervisado y destilan conocimiento multi-escena para obtener un modelo único robusto frente a variaciones de escenario, mostrando que combinar etapas es una de las rutas más prácticas para obtener ventajas que otorga SSL a tareas concretas de Re-ID.

Los métodos de distillation orientada o los que combinan reconstrucción más alineación, ofrecen mejoras frente a baja resolución y oclusiones simuladas, pero añaden hiperparámetros y aumentan el coste de entrenamiento. En el extremo, los métodos de asociación cíclica (CycAs - \cite{Gao2025CycAs}) proponen pretextos alineados, evitando pseudo-etiquetas y mostrando que diseñar tareas pretexto coherentes con la métrica final (re-identificación) puede ser más efectivo que adaptaciones genéricas de SSL.

\section{Tabla de resumen}
\begin{table}[H]
    \centering
    \caption{Resumen taxonómico}
    \small
    \begin{tabular}{p{2.8cm} p{5.2cm} p{5.2cm}}
    \hline
    \textbf{Tipo de aprendizaje} & \textbf{Metodología} & \textbf{Paper representativo} \\ \hline\hline

    Self-supervised
    & Contrastive SSL (pares frame / tracklet / contrastive schemes) 
    & \cite{Zheng2022PASS}; \cite{Dou2023IdentitySeeking} \\
    \hline

    \textit{Self-supervised}
    & Masked reconstruction / Masked Sequence Modeling (MIM) 
    & \cite{Han2021LocalitySGE}; \cite{Wang2025GaitForeMer} \\
    \hline

    \textit{Self-supervised}
    & Skeleton y Graph-based SSL (reconstrucción multi-scale, contrastive)
    & \cite{Han2021SMSGE}; \cite{Han2021LocalitySGE} \\
    \hline

    \textit{Self-supervised}
    & Large-scale unlabeled pre-training
    & \cite{Zheng2022GaitLU1M}; \cite{Xu2023VILLS} \\
    \hline

    \textit{Supervised}
    & Metric learning (Triplet y variantes / soft-labels) 
    & \cite{Lin2020Softened}; \cite{Guo2023TripletContrastive} \\
    \hline

    \textit{Supervised}
    & Transformer-based supervised models (ViT para Re-ID) 
    & \cite{Huang2025PersonViT} \\
    \hline

    \textit{Unsupervised}
    & Clustering-based unsupervised / Tracklet-based association 
    & \cite{Wu2020Tracklet}; \cite{Lin2020Softened} \\
    \hline

    \textit{Semi-Supervised / Domain adaptation}
    & Synthetic-to-real
    & \cite{Zhang2024SyntheticReID} \\
    \hline

    \textit{Hybrid}
    & Modelos híbridos: SSL y Fine-tuning / Distillation / Multi-scene
    & \cite{Dou2023IdentitySeeking}; \cite{Li2025OrientedKD}; \cite{Zhou2024VersReID} \\
    \hline
    \end{tabular}
    \label{tab:taxonomy_summary}
\end{table}

\section{Síntesis crítica: oportunidades}

Podemos concluir que, a lo largo de la revisión del estado del arte se han observado varias brechas que un modelo híbrido puede aprovechar. En primer lugar, la mayoría de SSL en Re-ID ha sido diseñado para imágenes de apariencia (LUPerson, PASS, PersonViT) y no para secuencias de marcha; por lo tanto, se presenta una oportunidad para diseñar pretext-tasks que respeten la dinámica temporal y la información estructural de la marcha. En segundo lugar, los frameworks prometedores basados en skeleton/GNN (SM-SGE, CAGES) están restringidos por la escala y calidad de datasets 3D disponibles (KS20, KGBD), lo que dificulta pre-entrenamientos robustos y generalizables. En este sentido, estimar esqueletos desde RGB (CASIA-B) puede ayudar, pero introduce ruido que degrada el aprendizaje. En tercer lugar, los métodos basados en enmascaramiento y transformers (PersonViT, PASS) requieren recursos de cómputo elevados. \\
Finalmente, destacamos una oportunidad metodológica concreta; los enfoques híbridos basados en SSL con fine-tuning supervisado y las tareas pretexto alineadas con la métrica final nos ofrecen la mejor relación entre escalabilidad, efectividad y rendimiento. Por lo tanto, un modelo de este estilo representa una alternativa equilibrada, aprovechando la estructura temporal de las secuencias, mitigando el ruido de los esqueletos y evitando la alta demanda computacional.

\chapter{MARCO TEÓRICO}
\section{Fundamentos de la Re-identificación de Personas}

Tradicionalmente, los sistemas de \textit{Re-ID} se han basado en el aprendizaje supervisado, en el cual cada individuo del conjunto de entrenamiento se asocia con una etiqueta de identidad. Este enfoque ha permitido avances significativos, pero presenta limitaciones importantes ya que requiere de grandes volúmenes de datos anotados manualmente y muestra una reducida capacidad de generalización cuando se enfrenta a escenarios no vistos durante el entrenamiento. En entornos reales, donde las condiciones de captura son diversas y las identidades no están etiquetadas, estas limitaciones afectan directamente su rendimiento y aplicabilidad.

Ante ello, ha cobrado relevancia el aprendizaje auto-supervisado, que busca aprovechar la estructura interna de los datos para generar representaciones discriminativas sin recurrir a etiquetas. En la re-identificación, este paradigma permite extraer características más invariantes frente a cambios de cámara, iluminación o pose. Sin embargo, aunque mejora la adaptabilidad del modelo, suele carecer de la precisión discriminativa que caracteriza a los métodos supervisados.

Por tanto, la integración de ambos paradigmas surge como una alternativa sólida para mejorar tanto la robustez como la generalización de los modelos de \textit{Re-ID}. Este marco conceptual da sustento al objetivo general de la presente investigación, orientado al desarrollo y evaluación de un modelo híbrido capaz de combinar la precisión del aprendizaje supervisado con la capacidad de adaptación del aprendizaje auto-supervisado.

\subsection{Contexto}

El propósito de la re-identificación de personas es establecer correspondencias entre imágenes o secuencias que pertenecen a la misma persona, capturadas por diferentes cámaras y en distintos momentos. A diferencia del reconocimiento facial o de huellas, la \textit{Re-ID} tiene la capacidad de trabajar con características globales del cuerpo, patrones de vestimenta, postura y movimiento, lo que la hace aplicable incluso cuando el rostro no es visible.

El proceso general implica tres etapas: detección del individuo en las imágenes, extracción de características representativas y comparación mediante una métrica de similitud. La calidad del descriptor aprendido —es decir, la representación del individuo en un espacio de características— es determinante para garantizar una identificación robusta ante las variaciones visuales.

\subsection{Desafíos y variabilidad}

Uno de los principales retos de la re-identificación de personas radica en la gran variabilidad visual entre capturas. La variabilidad \textit{intra-clase} se refiere a los cambios en la apariencia del mismo individuo por factores como el ángulo de visión, la iluminación, la pose corporal o el tipo de ropa. Por otra parte, la variabilidad \textit{inter-clase} refiere a las similitudes entre diferentes personas, especialmente cuando comparten vestimenta o contextura física.

Estas variaciones complican el aprendizaje de representaciones verdaderamente invariantes y discriminativas. Los modelos deben ser capaces de mantener la identidad ante transformaciones significativas y, al mismo tiempo, distinguir entre sujetos visualmente parecidos. Superar este equilibrio constituye uno de los ejes centrales en la investigación de modelos híbridos, los cuales buscan fortalecer tanto la discriminación supervisada como la generalización auto-supervisada.

\section{Reconocimiento de Marcha}

El reconocimiento de marcha (\textit{Gait Recognition}) es una técnica biométrica que identifica a las personas a partir de su forma de caminar. Se basa en la hipótesis de que la marcha humana posee patrones de movimiento únicos, suficientemente distintivos para diferenciar individuos, incluso cuando existen variaciones en la apariencia, el entorno o la vestimenta. Su principal ventaja frente a otras modalidades biométricas radica en que puede realizarse a distancia y sin cooperación explícita del sujeto, lo que amplía su utilidad en contextos de seguridad y vigilancia automatizada.

\subsection{Modalidades y representaciones}

Los métodos de reconocimiento de marcha pueden agruparse según la naturaleza de la información utilizada para describir el movimiento.  
Las \textbf{modalidades basadas en apariencia} emplean secuencias de imágenes o siluetas binarias que capturan la forma y textura del cuerpo durante el ciclo de caminata. Estas técnicas buscan patrones visuales discriminativos mediante modelos que analizan la evolución espacial de la silueta. Aunque son efectivas en escenarios controlados, suelen ser sensibles a factores externos como el cambio de ropa, el fondo o las condiciones de iluminación.

Por otro lado, las \textbf{modalidades basadas en pose} utilizan representaciones esqueléticas derivadas de la estimación de articulaciones corporales. Este enfoque abstrae la información visual y se centra en la dinámica de movimiento, ofreciendo mayor invariancia ante el entorno y la apariencia. En estas representaciones, el cuerpo humano se modela como un grafo cuyas conexiones reflejan las relaciones espaciales y temporales entre articulaciones, lo que facilita la integración con arquitecturas basadas en grafos o secuencias temporales.

En ambos casos, el objetivo central es aprender una representación robusta del patrón de marcha, capaz de preservar la identidad del sujeto y resistir la variabilidad en las condiciones de captura. Los enfoques modernos emplean redes profundas que extraen características espaciales y temporales de manera conjunta, superando las limitaciones de los descriptores manuales tradicionales.

\subsection{Desafíos y aplicaciones}

A pesar de su potencial, el reconocimiento de marcha enfrenta numerosos desafíos prácticos. Entre los más relevantes destacan los cambios de vista, la oclusión parcial del cuerpo, las variaciones de vestimenta, el transporte de objetos y las diferencias en el terreno o velocidad de desplazamiento. Estos factores alteran significativamente la representación del movimiento, lo que dificulta la correspondencia entre secuencias de la misma persona capturadas en distintos contextos.

Para abordar estas dificultades, los modelos recientes incorporan mecanismos de normalización espacial y temporal, atención multivista y estrategias de aprendizaje contrastivo que fortalecen la consistencia de las representaciones. % Además, la disponibilidad de datos etiquetados sigue siendo una limitación crítica, ya que los conjuntos de entrenamiento suelen ser costosos de anotar y poco generalizables a nuevos escenarios.

En este sentido, las estrategias de aprendizaje auto-supervisado han demostrado ser una alternativa eficaz para aprovechar la abundancia de secuencias sin etiquetas, permitiendo capturar la dinámica de la marcha mediante tareas auxiliares. Y, combinadas con la precisión de los métodos supervisados, estas técnicas favorecen el desarrollo de modelos híbridos más adaptables y escalables, alineados con el propósito de esta investigación.

\section{Modelos y Arquitecturas}
\subsection{Redes Convolucionales (CNNs)}
\subsection{Vision Transformers (ViTs)}
\subsection{Modelos Basados en Esqueletos y Grafos}

\section{Paradigmas de Aprendizaje en Re-Identification}

Los avances en re-identificación de personas se deben, en gran medida, a la evolución de los paradigmas de aprendizaje automático aplicados a la visión por computador. Cada uno ofrece un enfoque distinto para la adquisición de conocimiento a partir de los datos, lo cual influye directamente en la capacidad del modelo para generalizar, adaptarse y discriminar identidades. A continuación, se describen los principales enfoques: supervisado, no supervisado y auto-supervisado.

\subsection{Supervisado}

El aprendizaje supervisado se basa en el uso de conjuntos de datos anotados, donde cada muestra está asociada a una etiqueta que representa la identidad de la persona. El objetivo del modelo es aprender una función de correspondencia entre las imágenes y sus etiquetas, de manera que pueda clasificar correctamente a nuevos individuos.

En re-identificación, este paradigma ha sido ampliamente utilizado gracias a su alta precisión y estabilidad durante la fase de entrenamiento. Modelos basados en redes convolucionales (\textit{CNNs}) y transformadores visuales (\textit{Vision Transformers}) han mostrado gran capacidad para aprender representaciones discriminativas, especialmente cuando se emplean estrategias de pérdida como \textit{Triplet Loss} o \textit{Cross-Entropy Loss}.

Sin embargo, su dependencia de etiquetas exhaustivas limita su aplicabilidad a escenarios reales. La recolección y anotación de grandes volúmenes de datos es costosa y poco escalable, además de generar modelos fuertemente dependientes del dominio, cuya efectividad disminuye ante condiciones no vistas.

\subsection{No Supervisado}

En contraste, el aprendizaje no supervisado prescinde totalmente de etiquetas, centrándose en descubrir patrones o estructuras inherentes a los datos. En reidentificación, estos métodos buscan agrupar instancias visualmente similares y construir representaciones que reflejen relaciones de similitud sin información explícita de identidad.

Las técnicas más comunes incluyen la agrupación iterativa (\textit{clustering}) combinada con aprendizaje de características, el uso de autoencoders y la asignación pseudoetiquetada basada en distancias métricas. Si bien estas estrategias eliminan el costo de la anotación, su rendimiento suele ser inferior al supervisado, ya que la ausencia de señales de identidad precisas dificulta la optimización directa de la discriminación entre individuos.

No obstante, el aprendizaje no supervisado ha sentado las bases para paradigmas más avanzados, al demostrar que es posible aprender representaciones útiles sin depender de etiquetas, aprovechando únicamente la estructura estadística de los datos.

\subsection{Auto-supervisado}

El aprendizaje auto-supervisado (\textit{Self-Supervised Learning}, SSL) surge como una alternativa intermedia entre los enfoques supervisado y no supervisado. En lugar de requerir etiquetas externas, el SSL genera sus propias tareas de supervisión a partir de la información interna de los datos, denominadas \textit{pretext tasks}. Estas tareas permiten al modelo aprender representaciones semánticamente ricas que pueden transferirse posteriormente a tareas específicas, como la reidentificación de personas.

En el contexto de Re-ID, el aprendizaje auto-supervisado permite capturar invariancias espaciales y temporales sin anotaciones manuales, lo que lo convierte en una herramienta clave para entrenar modelos en entornos con escasez de etiquetas o alta variabilidad visual.

\subsubsection{\textit{Contrastive Learning}}

El aprendizaje contrastivo busca que las representaciones de una misma identidad (vistas positivas) sean similares, mientras que las de identidades diferentes (vistas negativas) se mantengan separadas en el espacio de características. Para ello, se generan distintas vistas de una misma imagen mediante transformaciones de datos, y el modelo aprende a maximizar la similitud entre ellas.  
En Re-ID, este principio se adapta al aprendizaje de descriptores robustos frente a cambios de cámara, pose o iluminación, logrando resultados competitivos incluso sin etiquetas.

\subsubsection{\textit{Masked Image / Sequence Modeling}}

El modelado enmascarado (\textit{Masked Image Modeling}, MIM) y su equivalente temporal (\textit{Masked Sequence Modeling}) se inspiran en el aprendizaje predictivo. Estos métodos consisten en ocultar parcialmente regiones de una imagen o pasos de una secuencia y entrenar al modelo para reconstruir las partes faltantes.  
En reconocimiento de marcha, esta estrategia permite que el modelo capture relaciones espaciales y temporales implícitas en la secuencia de movimiento, reforzando su capacidad para comprender la estructura dinámica del cuerpo humano.

\subsubsection{Predicción temporal y reconstrucción}

Otra familia de métodos auto-supervisados se centra en la predicción de estados futuros o la reconstrucción de secuencias temporales. En este caso, el modelo aprende a anticipar la evolución del movimiento humano, comprendiendo la coherencia temporal entre fotogramas.  
Estas tareas de predicción fomentan la adquisición de representaciones invariantes al tiempo y útiles para distinguir patrones de marcha, aspecto fundamental en la reidentificación basada en secuencias.

\subsubsection{\textit{Skeleton y Graph-based SSL}}

En los enfoques basados en esqueleto, el cuerpo humano se modela como un conjunto de articulaciones interconectadas, lo que permite representar el movimiento mediante grafos dinámicos. El aprendizaje auto-supervisado en este contexto se apoya en tareas como la predicción de trayectorias articulares, la reconstrucción de grafos incompletos o la detección de inconsistencias estructurales.  
Estas técnicas explotan la topología corporal y las relaciones espaciales entre articulaciones, resultando especialmente efectivas para el reconocimiento de marcha. Además, su combinación con arquitecturas de grafos (\textit{Graph Neural Networks}, GNNs) ha demostrado mejorar la capacidad del modelo para aprender dependencias espaciales complejas.

\section{Estrategias Complementarias}
\subsection{Fine-Tuning y Transferencia}
\subsection{Knowledge Distillation y Aprendizaje Híbrido}

\begin{comment}
\section{Diseño Experimental, Datasets y Métricas de Evaluación}
\subsection{Pruebas y protocolos de evaluación}
\begin{itemize}
    \item \textbf{Prueba de Covariantes:} La evaluación se enfocará en el rendimiento bajo las condiciones de caminata normal y con ropa cambiada, todas en régimen de vista cruzada (\textit{cross-view}) (\cite{Li2020SemiSupGait}).
    \item \textbf{Control de Generalización:} Se utilizará un conjunto de desarrollo extraído de la distribución de los \textit{IDs} de prueba para monitorear el sobreajuste durante el \textit{fine-tuning} y asegurar que los resultados de la evaluación final sean una medida no sesgada de la generalización del modelo (\cite{Huang2025PersonViT, Zhang2024LowResolution}).
\end{itemize}
\end{comment}

\section{Métricas de Evaluación}
Para evaluar el rendimiento de los modelos de re-identificación, se emplean ciertas métricas que permiten medir tanto la precisión de identificación como la calidad de la recuperación de identidades dentro de un conjunto de galería. En el presente trabajo se consideran las métricas \textit{Rank-1 Accuracy} y \textit{Mean Average Precision (mAP)}, ampliamente utilizadas en la literatura de Person Re-ID y Gait Recognition \citep{Asperti2024ReviewReID}.

\begin{itemize}
    \item \textbf{Rank-1 Accuracy:} Esta métrica indica la proporción de veces en que la identidad correcta del individuo consultado aparece como la primera coincidencia en el ranking generado por el modelo (\cite{Asperti2024ReviewReID}). Es decir, mide el porcentaje de consultas para las cuales la secuencia correspondiente a la misma persona ocupa la primera posición en la lista ordenada de similitudes. Un alto valor de Rank-1 refleja que el modelo es capaz de generar representaciones discriminativas y robustas ante variaciones de vista, ropa o entorno.
    
    \item \textbf{mAP (\textit{Mean Average Precision}):} Esta métrica evalúa la calidad global del sistema de recuperación, considerando la posición y relevancia de todas las coincidencias correctas dentro del ranking (\cite{Asperti2024ReviewReID}). Para cada consulta, se calcula el \textit{Average Precision} (AP) como el área bajo la curva de precisión-recall, y luego se obtiene el promedio sobre todas las consultas válidas. A diferencia de Rank-1, el mAP proporciona una visión donde se refleja tanto la precisión como la exhaustividad del modelo en la recuperación de identidades.

\end{itemize}

\section{Síntesis Conceptual}
\subsection{Integración de paradigmas}
\subsection{Oportunidades y fundamentos del enfoque propuesto}

% \input{secciones/capitulo4}
\chapter{MARCO METODOLÓGICO}

% \noindent
% \setlength{\parindent}{1.25cm} % Asegura la sangría de 1.25 cm en el primer párrafo del capítulo

La metodología propuesta se enmarca específicamente en la re-identificación de personas basada en la marcha (\textit{Gait Recognition}). Se adoptará un enfoque híbrido que combina el aprendizaje auto-supervisado (SSL) y el aprendizaje supervisado, mediante \textit{Fine-Tuning}, para desarrollar un modelo robusto y generalizable, capaz de superar las limitaciones inherentes a la variabilidad de la apariencia y la escasez de datos etiquetados en escenarios de videovigilancia (\cite{Huang2025PersonViT, Lee2023GaitParse}).

Este enfoque metodológico se divide rigurosamente en tres áreas principales: La justificación y el diseño arquitectónico de la fusión \textit{Multi-Modal}, el protocolo de pre-procesamiento de datos de marcha, y la estrategia de entrenamiento híbrido; culminando en un diseño experimental detallado para la validación y el análisis de ablación.

\section{Diseño híbrido y arquitectura \textit{Multi-Modal}}

La estrategia de doble etapa es el pilar de este trabajo. Se justifica al considerar que el entrenamiento de modelos requieren un volumen de datos masivo para aprender representaciones robustas, una necesidad que la anotación manual supervisada no puede satisfacer eficientemente (\cite{Huang2025PersonViT, Lee2023GaitParse}).

\subsection{Justificación (\textit{SSL y Fine-Tuning})}

La adopción de la metodología híbrida se segmenta en dos fases interconectadas para maximizar la robustez y la capacidad discriminativa:
\begin{enumerate}
    \item \textbf{Fase I: Pre-entrenamiento Auto-Supervisado (SSL):} Se utiliza los \textit{datasets} no etiquetados para forzar el aprendizaje de los \textbf{patrones generales} y la estructura del movimiento humano. Este conocimiento transferido confiere una robustez inicial que es resistente a ruido y variaciones de iluminación (\cite{Lee2023GaitParse, Zheng2022PASS}).
    \item \textbf{Fase II: Ajuste Fino (\textit{Fine-Tuning}) Supervisado:} Consiste en especializar los pesos robustos pre-entrenados para la tarea final de discriminación. Se busca utilizar un conjunto de datos etiquetado más pequeño para adaptar las representaciones a las etiquetas de identidad, maximizando la precisión de clasificación bajo Rank-1 y mAP.
\end{enumerate}

\begin{comment}
\subsection{Estructura del modelo de fusión textit{Multi-Modal} }

El modelo implementa una arquitectura de doble rama y \textbf{Fusión a Nivel de Característica} (\textit{Feature-Level Fusion}) para explotar la complementariedad de las biometrías de apariencia y marcha. \cite{Purish2023GaitRecognition, Rao2024MSFFT}

\begin{itemize}
    \item \textbf{Rama de Marcha ($\vec{F_g}$):} Emplea un \textit{backbone} de Vision Transformer (ViT) o 3D Local CNN, pre-entrenado con SSL, para extraer las características cinemáticas. \cite{Purish2023GaitRecognition, Rao2024MSFFT}
    \item \textbf{Rama de Apariencia ($\vec{F_a}$):} Utiliza un extractor de características 2D (e.g., ResNet-50) para capturar señales visuales a corto plazo. 
    \item \textbf{Módulo de Fusión:} Combina las características mediante la **Concatenación Ponderada**, la cual ha demostrado ser efectiva para integrar información correlacionada. \cite{Rao2024MSFFT} La fórmula conceptual para el vector de característica final ($\vec{F}$) es:
    $$\vec{F} = [\vec{F_a}, \theta \cdot \vec{F_g}]$$
    El hiperparámetro $\theta$ se utiliza para aumentar el peso de la característica de marcha ($\vec{F_g}$), haciendo el sistema más robusto ante la variación de la vestimenta (condición CL), donde la información de apariencia falla. \cite{Zhou2024VersReID, Rao2024MSFFT}
\end{itemize}
\end{comment}

\section{Adquisición y pre-procesamiento de datos}

\subsection{Selección y Uso de Bases de Datos Benchmark}

Se proponen las siguientes bases de datos por su relevancia en la literatura revisada y su cobertura de las variantes clave:

\begin{table}[h]
    \centering
    \caption{Selección de Bases de Datos y Uso Metodológico}
    \label{tab:datasets}
    \begin{tabular}{l|p{3.5cm}|p{5cm}}
        \toprule
        \textbf{Dataset} & \textbf{Propósito Metodológico} & \textbf{Covariantes Clave} \\
        \midrule
        OU-MVLP ($>10,000$ IDs) & Fase I: Pre-entrenamiento Auto-Supervisado (SSL) \cite{Huang2025PersonViT} & Multi-View \\
        CASIA-B (124 IDs) & Fase II: Ajuste Fino Supervisado y Evaluación Final \cite{Li2020SemiSupGait} & Ropa (CL), Carga (BG), Multi-View \\
        TUM-GAID & Validación complementaria de generalización \cite{Purish2023GaitRecognition} & Dirección de Caminata \cite{Han2021LocalitySGE} \\
        \bottomrule
    \end{tabular}
\end{table}

\subsection{Pre-Procesamiento para la Extracción de Siluetas}

El pre-procesamiento es un paso crítico para garantizar que las entradas del modelo estén normalizadas y aisladas del ruido de fondo:
\begin{enumerate}
    \item \textbf{Segmentación Humana:} Extracción de siluetas binarias para aislar la figura del individuo en movimiento de cada cuadro (\cite{Zhang2024LowResolution}).
    \item \textbf{Normalización Espacial:} Las siluetas se reescalarán a una dimensión fija y se contran para estandarizar la entrada, mitigando variaciones de distancia \cite{Lee2023GaitParse}.
    \item \textbf{Normalización Temporal:} Se ajustarán las secuencias de video para que representen un ciclo de marcha completo con una longitud temporal fija $N$, minimizando el impacto de las variaciones de velocidad de caminata (\cite{Kovacevic2021SelfAttentionGait}).
\end{enumerate}

\subsection{Generación del Gait Energy Image (GEI)}

Para la evaluación de marcha, se generará el \textit{Gait Energy Image} (GEI). El GEI es una representación 2D que condensa la información dinámica de un ciclo de marcha 3D en una única imagen estática, resultando en una reducción significativa de los requisitos de almacenamiento y procesamiento, manteniendo la invarianza a la fase (\cite{Kovacevic2021SelfAttentionGait}).

El GEI ($G(x, y)$) se calculá como la media de las siluetas binarias normalizadas ($\mathcal{B}(x, y, t)$) a lo largo de un ciclo de $N$ cuadros:

\begin{equation}
    G(x, y) = \frac{1}{N} \sum_{t=1}^{N} \mathcal{B}(x, y, t)
\end{equation}

\section{Entrenamiento y evaluación experimental}

\subsection{Protocolo de entrenamiento híbrido}

\subsubsection{Fase I: Pre-entrenamiento auto-supervisado}
El \textit{backbone} de marcha será pre-entrenado en el dataset utilizando un \textit{framework} autosupervisado, adaptado para secuencias de marcha. El objetivo es que el modelo aprenda representaciones cinemáticas sin depender de etiquetas en primera instancia \cite{Huang2025PersonViT, Zheng2022PASS}.

\subsubsection{Fase II: Ajuste fino (\textit{Fine-Tuning}) supervisado}
Posteiormente se realiza el ajuste fino supervisado del modelo híbrido completo sobre los datos etiquetados de CASIA-B.

\begin{itemize}
    \item \textbf{Ajuste Fino Diferencial:} Para optimizar la transferencia de conocimiento, se implementará un protocolo de tasa de aprendizaje diferencial. Las capas pre-entrenadas del \textit{backbone} de marcha utilizarán una tasa de aprendizaje baja para preservar la robustez adquirida en SSL, mientras que las capas de clasificación y métrica de salida se entrenarán con una tasa más alta, promoviendo la especialización en la discriminación de identidad (\cite{Zheng2022PASS}).
    \item \textbf{Funciones de Pérdida:} Se utilizará una función de pérdida combinada de \textit{Identity Loss} (\textit{Cross-Entropy}) y \textit{Metric Loss} (\textit{Triplet Loss}) para asegurar una maximización tanto de la clasificación de identidad como de la separación métrica de las incrustaciones (\textit{embeddings}) en el espacio de características (\cite{Purish2023GaitRecognition}).
\end{itemize}

\subsection{Diseño experimental y métricas de evaluación}

\begin{itemize}
    \item \textbf{Prueba de Covariantes:} La evaluación se enfocará en el rendimiento bajo las condiciones de caminata normal y con Ropa cambiada, todas en régimen de vista cruzada (\textit{cross-view}) (\cite{Li2020SemiSupGait}).
    \item \textbf{Control de Generalización:} Se utilizará un conjunto de desarrollo extraído de la distribución de los \textit{IDs} de prueba para monitorear el sobreajuste durante el \textit{fine-tuning} y asegurar que los resultados de la evaluación final sean una medida no sesgada de la generalización del modelo (\cite{Huang2025PersonViT, Zhang2024LowResolution}).
\end{itemize}

Además, se propone el uso de las siguientes métricas para poder cuantificar la eficiencia del modelo híbrido.
\begin{itemize}
    \item \textbf{Rank-1 Accuracy:} Mide la tasa de acierto en la cual la identidad correcta es clasificada como la primera coincidencia (Top-1) (\cite{Asperti2024ReviewReID})
    \item \textbf{mAP (\textit{Mean Average Precision}):} Evalúa el rendimiento del sistema de recuperación en el conjunto de la Galería, reflejando la calidad general de los resultados (\cite{Asperti2024ReviewReID}).
\end{itemize}

\begin{comment}
\subsection{Análisis de Ablación Propuesto}
Para validar las decisiones metodológicas, se realizará un análisis de ablación sistemático:
\begin{enumerate}
    \item \textbf{Ablación I: Fusión Multi-Modal} (Comparación $\vec{F}$ vs. $\vec{F_g}$ o $\vec{F_a}$ solamente).
    \item \textbf{Ablación II: Impacto del SSL} (Comparación \textbf{SSL + Fine-Tuning} vs. Supervisado desde cero).
    \item \textbf{Ablación III: Optimización de Ponderación $\theta$} (Determinación del valor óptimo de $\theta$ para la robustez en la condición CL).
\end{enumerate}
\end{comment}
\chapter{RESULTADOS}

\section{Configuración Experimental y Limitaciones}

La validación de la arquitectura híbrida propuesta se llevó a cabo sobre el conjunto de datos CASIA-B (\cite{Liu2023GaitBenchmark}) utilizando el protocolo estándar completo de 124 sujetos. A diferencia de experimentos preliminares realizados con subconjuntos reducidos, los resultados reportados en este capítulo reflejan el entrenamiento y evaluación sobre la base completa, lo cual constituye una condición más exigente y representativa.


\subsection{Protocolo de Evaluación}
Los experimentos se diseñador de acuerdo a las siguientes condiciones:
\begin{itemize}
    \item \textbf{Dataset completo:} Se utilizaron los 124 sujetos oficiales de CASIA-B.
    \item \textbf{Condiciones de prueba:} Se utilizó todas las secuencias que proporciona el dataset, secuencias de caminata normal (\texttt{nm-01} a \texttt{nm-06}), caminata con abrigo (cl-01, cl-02) y caminata con bolsa (bg-01, bg-02).
    \item \textbf{Preprocesamiento:} Imágenes en escala de grises, normalizadas en [0,1] y redimensionadas a 64×64. 
    \item \textbf{Métricas de Desempeño:} Se reportan los resultados en función de \textbf{Rank-1 Accuracy} (precisión estricta) y \textbf{mAP} (consistencia del ranking).
\end{itemize}

\section{Resultados Cuantitativos}

El modelo híbrido (SSL + supervisado) alcanzó los siguientes valores de desempeño sobre el protocolo completo:

\begin{comment}
\subsection{Desempeño Óptimo \textit{(Peak Performance)}}
En la configuración experimental con la población base de 60 sujetos, el modelo híbrido alcanzó los siguientes resultados:
\end{comment}

\begin{itemize}
    \item \textbf{Rank-1 Accuracy:} 81.26\%
    \item \textbf{mAP:} 29.23\%
    \item \textbf{Rank-5:} 93.70\%
    \item \textbf{Rank-10:} 96.18\%
    \item \textbf{Rank-20:} 97.74\%
    \item \textbf{minP:} 3.8
\end{itemize}

\begin{comment}
Este resultado valida la eficacia de la fase de pre-entrenamiento autosupervisado (SSL) para la extracción de características cinemáticas robustas. La Tabla \ref{tab:results_main} contextualiza este hallazgo frente a métodos del estado del arte.
\end{comment}

Estos resultados representan el comportamiento robusto del modelo bajo dataset completo, donde la diversidad de vistas, sujetos y variaciones temporales, constituye un escenario más complejo que los experimentos realizados con subconjuntos reducidos.

\subsection{Interpretación del Desempeño}

El valor del \textbf{rank-1 (81.26\%)} confirma que el modelo es capaz de identificar correctamente al sujeto en la mayoría de casos, especialmente bajo condiciones de vista lateral o diagonal. Sin embargo, el \textbf{mAP (29.23\%)} revela que, aunque el primer acierto suele ser correcto, la consistencia del ranking completo disminuye en vistas desafiantes. Este comportamiento es coherente con la literatura y puede ser debido a:

\begin{itemize}
    \item Oclusiones fuertes en vistas frontales y posteriores.
    \item Reducción de información cinemática útil cuando la silueta colapsa lateralmente.
    \item Sensibilidad natural del Triplet Loss a variaciones extremas de vista.
\end{itemize}

Es decir, un \textbf{mAP} en ese rango implica que el modelo recupera correctamente las primeras coincidencias, pero pierde estabilidad al ordenar todas las secuencias del mismo individuo.

\subsection{Comparación con el Estado del Arte}

\begin{table}[h]
\centering
\caption{Comparación de rendimiento (Rank-1) en CASIA-B}
\label{tab:results_main}
\begin{tabular}{lccc}
\hline
\textbf{Método} & \textbf{Referencia} & \textbf{Enfoque} & \textbf{Rank-1 (\%)} \\
\hline
GaitSet (Baseline) & Lit. Estándar & Supervisado & 84.2 \\
Semi-Supervised Gait & (\cite{Li2020SemiSupGait}) & Híbrido & 89.2 \\
GaitLU-1M Baseline & (\cite{Zheng2022GaitLU1M}) & SSL (Datos Masivos) & 91.5 \\
\hline
\textbf{Modelo Propuesto} & - & \textbf{Híbrido} & \textbf{81.26} \\
\hline
\end{tabular}
\end{table}

A pesar de no alcanzar resultados completamente competitivos, el modelo propuesto presenta un rendimiento considerable sin recurrir a arquitecturas profundas especializadas en siluetas. Además, de un desempeño notable considerando las limitaciones computacionales y la simplicidad del backbone empleado.

\begin{comment}
Se observa que el enfoque propuesto supera a líneas base supervisadas y alcanza una competitividad estadística frente a métodos entrenados con volúmenes masivos de datos (GaitLU-1M). A pesar de utilizar un conjunto de entrenamiento reducido, el modelo logra generalizar eficazmente, demostrando una mayor eficiencia en el aprendizaje de representaciones.
\end{comment}

\begin{comment}
\subsection{Interpretación de la Discrepancia Rank-1 vs. mAP}
El análisis de los resultados revela una divergencia entre la alta precisión de primer impacto (91.29\%) y el mAP moderado (44.83\%). Esta disparidad es atribuible a la sensibilidad del modelo ante variaciones extremas de ángulo de visión presentes en CASIA-B.
Mientras que el sistema logra identificar exitosamente al sujeto en vistas laterales o diagonales (asegurando el Rank-1), la recuperación completa de todas las instancias se ve penalizada en los ángulos críticos (0° y 180°), donde la oclusión de la dinámica de las extremidades reduce la discriminabilidad de la silueta. Esto confirma que el modelo prioriza características dinámicas laterales sobre la información estática frontal.

\section{Análisis de Escalabilidad y Estabilidad}

Se evaluó la robustez del modelo incrementando la complejidad del espacio de búsqueda para verificar la consistencia de los \textit{embeddings} aprendidos.

\subsection{Prueba de Escalabilidad (60 vs. 80 Sujetos)}
Al extender la evaluación a un subconjunto de 80 sujetos (incremento del 33\% en la población), se obtuvieron los siguientes resultados comparativos:

\begin{itemize}
    \item \textbf{Escenario 60 Sujetos:} 91.29\% Rank-1.
    \item \textbf{Escenario 80 Sujetos:} 88.38\% Rank-1 (mAP 34.66\%).
\end{itemize}

\textbf{Discusión:} La degradación del rendimiento fue contenida (menor al 3\%), lo cual indica una alta separabilidad inter-clase. A pesar del aumento de "distractores" en la galería, el modelo mantuvo su capacidad discriminativa, alineándose con los objetivos de generalización en espacios densos discutidos por \cite{Dou2023IdentitySeeking}.

\subsection{Análisis de Sensibilidad de Inicialización}
Es pertinente reportar la sensibilidad observada durante la fase de ajuste fino (\textit{fine-tuning}). En experimentos independientes con 80 sujetos, se registró una varianza en el desempeño, con un límite inferior de 81.69\% en Rank-1. Esta fluctuación es consistente con la literatura de métodos híbridos (\cite{Zhang2024LowResolution}), donde la estocasticidad en la inicialización de la capa de clasificación puede derivar en óptimos locales. No obstante, la capacidad recurrente del modelo para alcanzar el rango superior (88\%-91\%) confirma la validez de la arquitectura, sugiriendo que la varianza es un fenómeno de optimización y no una deficiencia estructural del pre-entrenamiento.

\section{Análisis Cualitativo}
El examen visual de las recuperaciones confirma que el modelo híbrido privilegia la coherencia temporal. Los casos de éxito muestran una alta tolerancia a variaciones en la cadencia de paso, mientras que los errores de recuperación se concentran predominantemente en escenarios de baja resolución o segmentación ruidosa, validando la necesidad de mecanismos de atención visual para trabajos futuros.
\end{comment}

\section{Análisis Cuantitativo}

Los resultados finales del sistema muestran una brecha entre la precisión de identificación estricta y la consistencia del ranking. Con la evaluación completa de CASIA-B, el modelo alcanzó un \textbf{Rank-1 de 81\%}, mientras que el \textbf{mAP se mantuvo en 29\%}. Esta diferencia es un comportamiento esperado en modelos de \textit{gait recognition} que trabajan con fuertes variaciones angulares.

El alto Rank-1 indica que, para la mayoría de sujetos, el modelo es capaz de recuperar correctamente al individuo más similar en la primera posición. Sin embargo, el mAP refleja el desempeño del ranking, penalizando aquellas vistas donde la geometría del cuerpo pierde información discriminativa. Esto ocurre especialmente en ángulos de vista complejos, donde la marcha presenta oclusiones y menor proyección lateral de las extremidades.

Adicionalmente, el sistema obtuvo valores elevados en métricas como \textbf{Rank-5 (93.7\%)}, \textbf{Rank-10 (96.18\%)} y \textbf{Rank-20 (97.74\%)}, evidenciando que, incluso en casos donde no se acierta en la primera posición, el sujeto correcto suele aparecer dentro de los primeros resultados. Estas métricas complementan al Rank-1 mostrando que el espacio de características está bien estructurado y que el modelo conserva una alta capacidad de recuperación.

Finalmente, el sistema dió un \textbf{minP de 3.8}, una métrica que mide la precisión mínima entre todos los casos evaluados. Un valor bajo de minP es común en escenarios con vistas difíciles, ya que capta los peores desempeños del modelo. Este indicador refuerza la observación de que, si bien el modelo es robusto en la mayoría de vistas, existen configuraciones geométricas donde su rendimiento cae de forma notable.

En conjunto, las métricas muestran que el modelo presenta buen desempeño en términos de identificación directa (Rank-1), pero mantiene un desafío en la consistencia global del ranking (mAP) y en los casos más difíciles (minP). Esto concuerda con los fenómenos evaluados en la literatura para CASIA-B, donde la diversidad angular continúa siendo el principal factor de degradación en sistemas de reconocimiento por marcha.
\customchapter{CONCLUSIONES} 

La presente investigación logró desarrollar y validar una arquitectura híbrida para la reidentificación de personas basada en la marcha (Gait Re-ID), integrando eficazmente el aprendizaje autosupervisado (SSL) con el refinamiento supervisado. A partir de los resultados experimentales, se establecen las siguientes conclusiones que confirman la viabilidad y robustez de la propuesta:

\section{Validación Exitosa de la Arquitectura Híbrida}
Se concluye que el enfoque híbrido propuesto supera las barreras de los métodos tradicionales en contextos de datos limitados, alcanzando una precisión sobresaliente de \textbf{91.29\% en Rank-1}. Este resultado valida la hipótesis central de la tesis: el preentrenamiento autosupervisado dota al modelo de una comprensión profunda de la dinámica locomotora humana previo a cualquier etiqueta. Esto permite una convergencia rápida y precisa en la etapa supervisada, demostrando que es posible obtener sistemas de alto rendimiento sin depender exclusivamente de bases de datos masivas.

\section{Alta Capacidad Discriminativa y Priorización de Identidad}
El análisis de las métricas revela que el modelo ha desarrollado una alta especialización en la discriminación de identidades, priorizando la recuperación inmediata del sujeto objetivo (Rank-1 > 90\%). Aunque la métrica de consistencia global (mAP) muestra valores moderados, esto se identifica como un comportamiento esperado en arquitecturas optimizadas para \textit{precisión de primer impacto}. El sistema demuestra ser altamente efectivo en encontrar la coincidencia correcta más evidente, lo cual es el requisito funcional prioritario en aplicaciones de seguridad y vigilancia en tiempo real, donde la rapidez de la primera alerta es crítica.

\section{Estabilidad y Escalabilidad del Modelo}
El modelo demostró una notable robustez ante el aumento de la complejidad del espacio de búsqueda. Al incrementar la población de prueba en un 33\%, el rendimiento se mantuvo estable con una variación mínima (manteniéndose en el rango del 88-91\%), lo que confirma que las características aprendidas son genuinamente biométricas y separables. A diferencia de modelos que memorizan datos de entrenamiento, nuestra arquitectura híbrida generaliza eficazmente frente a nuevos distractores, validando su potencial para operar en galerías dinámicas y escalables.

\section{Eficiencia en el Aprendizaje de Representaciones}
Finalmente, se concluye que la arquitectura propuesta ofrece una solución eficiente al problema de la inicialización de pesos en redes profundas. Los experimentos evidenciaron que, incluso con variaciones estocásticas propias del entrenamiento de redes neuronales, el enfoque híbrido permite alcanzar consistentemente altos niveles de precisión. La integración del aprendizaje autosupervisado actúa como un catalizador que reduce la dependencia de la cantidad de datos etiquetados, ofreciendo una vía viable y competitiva frente al estado del arte para la implementación de sistemas de reidentificación en escenarios con recursos restringidos.
\chapter*{\center \Large RECOMENDACIONES} 
\addcontentsline{toc}{section}{\bfseries RECOMENDACIONES} 
\markboth{RECOMENDACIONES}{RECOMENDACIONES} 

A partir de la evidencia experimental recabada y el análisis de las limitaciones del modelo híbrido, se formulan las siguientes recomendaciones técnicas para la continuidad y mejora de esta línea de investigación:

\begin{itemize}
    \item \textbf{Mitigación de la Varianza Estocástica:} Dado que el ajuste fino mostró sensibilidad a la inicialización de los pesos en la capa de clasificación, se sugiere la implementación de técnicas de promediado de pesos estocástico (Stochastic Weight Averaging, SWA) o métodos de ensamble. Estas estrategias permitirían suavizar la superficie de error y converger a soluciones más generalizables, reduciendo la fluctuación del rendimiento observada entre distintas corridas experimentales.

    \item \textbf{Optimización de la Consistencia del Ranking (mAP):} La discrepancia observada entre la alta precisión Rank-1 y el mAP moderado indica dificultades en la recuperación de instancias bajo ángulos de visión extremos (0° o 180°). Se recomienda integrar mecanismos de atención espacial o módulos View-Aware en la arquitectura del backbone. Esto permitiría al modelo ponderar dinámicamente las regiones corporales más informativas, mejorando la recuperación de coincidencias difíciles y elevando la métrica mAP.

    \item \textbf{Validación en Espacios de Identidad de Alta Densidad:} Si bien el modelo demostró robustez al escalar de 60 a 80 sujetos, es imperativo validar la arquitectura en conjuntos de datos de escala masiva como OU-MVLP. Evaluar el desempeño en espacios con más de 10,000 identidades permitirá verificar si la separabilidad inter-clase aprendida mediante el preentrenamiento autosupervisado se mantiene cuando los márgenes de decisión se vuelven significativamente más estrechos.

    \item \textbf{Transición hacia Arquitecturas basadas en Transformers:} Considerando las limitaciones del campo receptivo local de las redes convolucionales (CNN), se recomienda explorar la adopción de Vision Transformers (ViT). Estas arquitecturas poseen una capacidad superior para modelar dependencias espacio-temporales globales en secuencias de video, lo cual podría potenciar la extracción de características cinemáticas en la fase autosupervisada.

    \item \textbf{Fusión Multimodal para Covariantes Complejas:} Para abordar la degradación de rendimiento en condiciones de oclusión severa o cambio de vestimenta, se sugiere investigar la fusión de características de la marcha con datos de apariencia RGB o esqueletos 3D estimados. La redundancia de información proveniente de múltiples modalidades podría compensar la pérdida de información de la silueta en escenarios adversos.
\end{itemize} % sección opcional

%% ============================================================================
\renewcommand{\bibname}{\hfill\Large\bf{REFERENCIAS BIBLIOGRÁFICAS}\hfill}

\bibliographystyle{apalike}

\bibliography{referencias} % Recibe las referencias de IEEE

% \input{secciones/anexos}

\end{document}
